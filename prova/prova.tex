\documentclass{article}
% \documentclass{ujarticle}
% \documentclass{jsarticle}

\usepackage[a4paper]{geometry}

\usepackage[brazil]{babel}
\usepackage[T1]{fontenc}
\usepackage{indentfirst}

\usepackage{hyperref}

\usepackage{multicol}

% \usepackage{CJKutf8}
% \newcommand{\jap}[1]{\begin{CJK*}{UTF8}{min}#1\end{CJK*}}
\newcommand{\jap}[1]{#1}
\newcommand{\jquote}[1]{「#1」}

\usepackage{xeCJK}
\setCJKmainfont{IPAMincho}[
    UprightFont=*,
    BoldFont=IPAPGothic,
    ItalicFont=*,
    BoldItalicFont=IPAPGothic,
]
\setCJKmonofont{IPAPGothic}

\newcommand{\todo}[1]{\textbf{TODO:} \textit{#1}}
\newcommand{\todoinl}[1]{#1}

\newcommand{\keyword}[1]{\textbf{#1}}

\newcommand{\phone}[1]{\slash\textit{#1}\slash}

\newcommand{\dekassegui}{\jap{出稼ぎ}}
\newcommand{\koroniago}{\jap{コロニア語}}

\newcommand{\koronia}{\keyword{koronia-go}}

\newcommand{\dre}{119025937}
\newcommand{\hashi}{Daniel Kiyoshi Hashimoto Vouzella de Andrade}

\newcommand{\mytitle}{Fundamentos da Linguística 2023.1 - 7h30}
\newcommand{\myauthor}{\hashi{} \\
    Aluno do Curso de Ciências da Computação \\
    DRE N\textdegree{} \dre{}
}
\date{}
\newcommand{\makecapa}{
    \newpage
    \hfill\par\vfill

    \begin{center}
        \huge
        \mytitle{}
    \end{center}

    \vfill

    \begin{center}
        \Large
        \myauthor{}
    \end{center}

    \vfill
    \vfill
    \begin{multicols}{3}
        \par\hfill \columnbreak

        \par\hfill \columnbreak

        Trabalho entregue à
        Prof. Maria Carlota Rosa,
        na disciplina
        Fundamentos da Linguística,
        turma 5727,
        código LEF140.
    \end{multicols}

    \vfill

    \begin{center}
        Faculdade de Letras \\
        Rio de Janeiro, 29 de junho de 2023, \\
        1\textdegree{} semestre
    \end{center}

    \pagenumbering{gobble}
    \newpage
    \pagenumbering{arabic}
}

\begin{document}
\makecapa

\setcounter{section}{-1}
\section{Antes de tudo}

A palavra importada \keyword{dekassegui}
vem do japonês \jquote{\dekassegui{}}
que é usado para se referir a ação de uma pessoa
sair (significado do ``radical'' 出)
do seu local de origem,
antigamente uma província ou agora um país,
à procura de trabalho ou ganhar dinheiro
(significado do verbo 稼く).
Hoje em dia, essa palavra, quando referida a uma pessoa,
pode ser traduzida
como `imigrante trabalhador' ou `trabalhador nômade'.
Já aqui no Brasil, um \keyword{dekassegui brasileiro}
é um brasileiro que sai do Brasil para o Japão
buscando trabalhar,
geralmente apenas `dekassegui' é usado.

Logo no prefácio, o livro \cite{filhosdekasegi}
ressalta que as experiências dos migrantes são heterogêneas;
assim como suas características, que variam desde
a motivação, idade, conhecimento da língua e cultura,
quem vai junto
-- pode-se ir sozinho, com a família inteira ou
com apenas parte dela --
até a legalidade da migração.
Considerando essa enorme variação entre os dekasseguis,
é inaceitável pensar que a situação de seus filhos
sejam homogêneas.

Por isso, não é razoável dizer
o que \emph{vai} acontecer,
as respostas podem ser diferentes para pessoas diferentes,
como foi relatado pelas filhas do sr. Ito, por exemplo;
ambas cresceram no Japão,
mas quando ``retornam'' ao Brasil
a caçula se adapta melhor e acaba aprendendo português,
enquanto a mais velha ainda tem grandes dificuldades
com a nova língua.
No lugar de dizer o que \emph{vai} acontecer,
serão descritas algumas situações
e, em seguida,
o que se espera que aconteça
a partir de cada uma delas.

A palavra \koronia{} vem de \jquote{\koroniago{}},
que é a junção de コロニア (\phone{koronia})
--- transliteração da palavra portuguesa `colônia'
(observe que o mais próximo do som de \phone{lo},
no japonês é o som de \phone{ro}) ---
com o sufixo 語 (nesse caso: \phone{go}),
usado para dizer a língua de um local,
por exemplo em \jquote{日本語}
(\phone{nihongo}, 日本 significa Japão)
ou \jquote{ポルトガル語}
(\phone{porutogarugo}, ポルトガル significa Portugal).
Dessa forma, \koronia{} teria como tradução
`a língua da Colônia'\footnotemark{}.
Essa língua, brevemente descrita
na seção 5.9 de \cite{viagemling},
é uma variedade do japonês
com forte presença de palavras \emph{aportuguesadas},
falado geralmente por
\keyword{nisseis} e \keyword{sanseis}
(descendentes de japoneses de segunda/terceira geração ---
filhos/netos)
que cresceram no Brasil.

\footnotetext{``Colônia'' aparece captalizada,
pois se referencia à
Colônia japonesa no Brasil,
diferente de uma colônia japonesa
em um outro lugar.}

Comparativamente,
\koronia{} se assemelha aos
pequeno ``dialeto'' que alguns jovens conversam entre si,
no qual se usa o português misturado
com estrangerismo vindo de jogos eletrônicos,
muitas vezes incompreensível para pessoas mais velhas.
São exemplos: \\
``Eu não \emph{tanko} ficar esse tempo todo na academia''; \\
``Ela \emph{givou} o namorado para ficar estudando para a prova''; \\
``O menino só \emph{spawnou} do meu lado; eu tomei um susto''; \\
A: ``Cadê o seu amigo?'' B: ``Ele \emph{foi de base}''. \\
Mesmo existindo palavras ou expressões inteiramente do português
que expressam o mesmo significado,
os falantes acabam preferindo as versões alternativas
por serem mais familiares.
Nesses contextos,
o verbo \emph{`tanko'} poderia ser substituído
por `aguento' ou `suporto'
\cite{tankaruol} \cite{tankarwikti};
\emph{`givou'} por `abriu mão [de estar com]';
\emph{`spawnou'} por `apareceu', `surgiu' ou \emph{`brotou'}
\cite{spawnardicioinf} \cite{spawnarwikti};
\emph{`foi de base'} por `foi embora',
`não está mais aqui' ou `está ocupado (fazendo outra coisa)'
\cite{irdebasehinative} \cite{irdebasewikti} \cite{irdebasedicioinf}.
Dessas \emph{gírias} a mais famosa é \emph{tankar}.
Observa-se que elas não são compartilhadas por todos os jovens,
da mesma forma que o \koronia{} não
é compartilhado por todos os descendentes nipônicos.

\section{Português como língua materna}

Considerando as crianças muito pequenas
que foram para o Japão antes de aprender a falar português,
podemos apontar dois casos
baseados na língua em que
os pais decidem se comunicar com a criança:
japonês e português.
O primeiro, torna-se bem natural para os pais,
já prevendo que seu filho vai interagir com
crianças e adultos japoneses,
muito provavelmente vai frequêntar uma escola japonesa e
assim por diante.
Com essa decisão,
não haveria um momento em que a criança
praticaria o português,
portanto ela nem aprenderia o português,
e assim o português não teria nenhuma chance de
ser sua \keyword{língua materna}.

Mesmo que a primeira escolha possa parecer muito atraente,
a segunda não é impossível:
muitos dekasseguis chegam ao Japão sem dominar o japonês,
ou mesmo que dominem podem planejar voltar para o Brasil
em um futuro próximo.
Nessa situação, o resultado vai depender bastante
de o que acontecer durante o desenvolvimento da criança.
Se crescer num ``ambiente japonês''
-- escolas, vizinhos e colegas falando em japonês --
apenas no ``melhor'' dos casos,
o português vai ter características de uma
\keyword{língua de herança} e
teria o japonês como outra \keyword{primeira língua (L1)}.

Também, pode crescer imersa num ``ambiente brasileiro'',
existem comunidades e até bairros cheios de brasileiros
\cite{japaoimigrantesbrasileiros} \cite{escolasbrjp}
\cite{regioesmaisbrjp}
\cite{videobairrojpcarla} \cite{videobairrojpfernando}
\cite{videocidadebr}.
Dessa forma, ainda há a pequena possibilidade de
ter o português como \keyword{língua materna}
enquanto sabe pouco ou nada de japonês.

Como um pequeno exemplo,
em \cite{videobairrojpcarla},
há fragmentos em que o filho compreende a mãe falando
em português mas responde em japonês.
Em um outro vídeo da mesma família \cite{videoramencarla},
percebe-se que o pai fala em japonês,
mas em nenhum desses dois vídeos
o filho falou algo em português,
dando a entender que sua situação
se parece mais com uma mistura do primeiro caso
com o segundo em um ambiente japonês.

\section{Koronia-go (\koroniago{}) e os filhos de dekasseguis}

As crianças que aparecem nas entrevistas
realizadas no estudo de Resstel \cite{filhosdekasegi}
tinham pouco ou nenhum conhecimento do português e
tinham contato majoritariamente, se não exclusivamente,
com falantes de japonês.
No Japão, elas não tinham motivo para falar ou saber \koronia{}.
Quando retornaram (ou vieram) ao Brasil,
todas elas tiveram que aprender português.
Lembrando que \koronia{} é uma língua marcada pela
mistura de palavras do português à gramática e língua japonesa
é inimaginável pensar que essas crianças conseguiriam falar
essa variedade do japonês.
Uma situação mais provável seria
que elas,
após aprenderem um pouco da língua,
falassem português
completando vocabulários desconhecidos
com palavras já conhecidas do japonês;
dado que a língua dominante delas é o japonês.

\newpage

\nocite{*}
\bibliographystyle{abntex2-num}
\bibliography{prova}

\end{document}
